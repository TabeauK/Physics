\documentclass{article}
\usepackage[utf8]{inputenc}
\usepackage{graphicx}


\title{Wyznaczanie stalej Planck'a}
\author{Dymitr Lubczyk }
\date{18 Maja 2020}

\begin{document}

\maketitle
\clearpage
\section{Czesc teoretyczna}
\subsection{Wstep}
W ponizszym sprawozdaniu zamierzam przedstawic przebieg doswiadczalnego wyznacznia stalej Planck'a. Bede korzystal z efektu fotoelektrycznego opisanego w 1905 roku przez Alberta Einsteina. Zjawisko to wyraza sie wzorem:
\[ h\nu = W+E_{k} \]
gdzie 

\begin{itemize}
  \item \(h\) - stala Planck'a
  \item \(\nu\) - czestotliwosc padajacego swiatla
  \item \(W\) - praca wyjscia (zalezna od materialu) 
  \item \(E_{k}\) - energia kinetyczna elektronu  
\end{itemize}
Zjawisko to pokazuje miedzy innymi zaleznosc miedzy czestotliwoscia, a energia fotonu. Co wiecej widzimy, ze energia ta rowna sie sumie pracy wyjscia oraz energii kinetycznej elektronu. Na podstawie tego mozemy skonstruawac doswiadcznenie majce na celu wyznacznie stalej Planck'a.
Wprowadzmy pojecie napiecia hamowania, zostanie ono oznaczone przez \(U_{h}\) i bedzie oznaczalo napiecie niezbedne, zeby dla danego ukladu i danej czestotliwosci swiatla prad przestal plynac. Majac na uwadze, ze w momenicie gdy w obwodzie przestaje plynac prad mamy \(E_{k}=eU_{h}\) pierwotny wzor mozemy przeksztalcic do:
\[ h\nu = W+eU_{h} \]
co daje liniowa zaleznosc miedzy czestotliwoscia padajacego swiatla, a napieciem hamowania i te zaleznosc bede staral sie wykorzystac, w celu wyznaczenia stalej Planck'a, ktora jest ukryta we wspolczynniku kierunkowym wspomnianej wyzej zaleznosci.
\subsection{Uklad doswiadczalny}
Uklad doswiadczalny sklada sie z katody i anody podlaczonych do zasilacza regulowanego z dokladnoscia do 10mV. Do uklady podlaczony jest rowniez amperamierz, w celu stwierdzenia, czy przez uklad plynie prad, jest to kluczowo do wyznaczenia \(U_{h}\). Co wiecej na uklad sklada sie lampa, z regulowana z dokladnoscia do 10nm dlugoscia fali emitowanego swiatla. Lampa ta oswietla katode w celu zainicjalizowana zjawiska fotoelektrycznego.
\clearpage
\section{Czesc doswiadczalna}
\subsection{Pomiary i niepewnosci}
\begin{table}[h!]
\centering
\begin{tabular}{|l|l|l|l|}
\hline
\(U_{h}[mV]\)   & \(\lambda[nm]\)   &\(f[THz]\) & \(u(f)[THz]\)         \\ \hline
1110    &410    &731    &178\\ \hline
970     &430    &697    &162\\ \hline
910     &440    &681    &154\\ \hline
780     &460    &652    &141\\ \hline
570     &500    &600    &120\\ \hline
380     &540    &550    &102\\ \hline
260     &570    &526    &92\\ \hline
150     &600    &500    &83\\ \hline

\end{tabular}
\caption{Napiecie hamowania, dlugosc fali oraz jej czestotliwosc wraz z niepewnoscia}
\end{table}
W powyzszych pomiarach przyjmujemy niepewnosc \(U_{h}\) jako \(10mV\), niepewnosc \(\lambda\) jako \(10nm\) oraz niepewnosc czestotliwosci jest wyznaczona jako  niepewnosc zlozona w zaleznosci od \(\lambda\), obliczona zgodnie ze wzorem
\[
u(f)=\frac{c\cdot u(\lambda)}{\lambda^2}
\]
\subsection{Wyznaczenie stalej Plancka}

\begin{figure}[h!]
\centering
\includegraphics[width=\textwidth]{Planck.JPG}
\caption{Wykres napiecia hamowania od czestotliwosci swiatla}
\end{figure}

Jak widac na powyzszym wykresie wspolczynik kieruknkowy prostej wynosi \(4.149\times\) \(10^{-15}\) [\(V \cdot s\)] oznaczmy go przez \(a\), wiemy ze:

\[
a=\frac{h}{e}
\]
wiec, zgodnie z powyzszym doswiadczeniem
\[
h=6,646\times 10^{-34} [J\cdot s] 
\]
z niedokladnoscia pomiarowa, sugerowana przez program Excel na poziomie \(0.0213\times 10^{-34} [J\cdot s] \)

\section{Podsumowanie}
Uwazam, ze efekt doswiadczenia jest bardzo satysfakcjonujacy, poniewaz udalo sie wyznaczyc stala Plancka z niebywala dokladnoscia, jednakze ma on wiele problemow, przede wszystkim doswiadcznie zostalo przeprowadzone na podstawie wynikow ostrzymanych z symulacji, a nie zebranych podczas faktycznego doswiadczenia, dodatkowo liczba zebranych probek jest wyjatkowo mala, co jest kolejnym brakiem. Uwazam, ze gdyby  bylo to mozlwie, bardzo ciekawym byloby odtworzenie tego doswiadczenia w fizycznym laboratorium. Tym bardziej, ze ukazuje ono znacznie wiecej niz wartosc stalej Plancka.
\end{document}

